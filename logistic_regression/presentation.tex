\documentclass{beamer}
\usetheme{Singapore}

\usepackage{url}
\usepackage{amsmath}
\usepackage{amsfonts}
\usepackage{amssymb}
\usepackage{breqn}
\usepackage{natbib}%,bibspacing}
\usepackage{graphicx}
\usepackage{bm}
%\usepackage{geometry}
\usepackage{float}
\usepackage{setspace}
\usepackage[english]{babel}
\usepackage[autostyle]{csquotes}

\title{Women and The Environment}
\subtitle{A Bayesian Logistic Regression Analysis}
\author{Kaylee Hodgson}
\institute{Brigham Young University}
\date{\today}

\begin{document}

\begin{frame}
\titlepage
\end{frame}

\begin{frame}
\frametitle{Introduction}
\begin{columns}
	\begin{column}{0.4\textwidth}
	\large
Subordination of women in the home seems to be indicative of a violent culture that does not treat its valuable assets well, including the environment.
	\end{column}
	\begin{column}{0.6\textwidth}
\begin{figure}
\includegraphics[width=6cm,height=3.5cm]{Climate}
\end{figure}
	\end{column}
\end{columns}
\bigskip
\bf{Purpose}: identify whether higher levels of gender inequality in the household are significantly tied to poor environmental wellbeing outcomes
\end{frame}

\begin{frame}
\frametitle{The Syndrome Scale}
Combines 11 indicators of the subordination of women in the home: 

\begin{itemize}
\tiny
\begin{columns}
	\begin{column}{0.48\textwidth}
	\item prevalence of patrilocal marriage
\item prevalence of brideprice or dowry
\item prevalence and legality of polygyny
\item presence of counsin marriage
\item age of marriage for girls
\item laws and practices surrounding women's property rights
	\end{column}
	\begin{column}{0.48\textwidth}
	\item presence of son preferences or sex ratio alteration
\item presence of inequity in family law/custom that favors males
\item overall level of violence against women in society
\item presence of societal sanction for femicide
\item whether there is legal exoneration for rapists who offer to marry their victims 
	\end{column}
\end{columns}
\end{itemize}
\begin{figure}
  \begin{centering}
  \includegraphics[height=4cm,trim={2cm 7cm 2cm 5.1cm},clip]{mapillust.pdf}
  \label{SynMap}
    \end{centering}
\end{figure}
\end{frame}

\begin{frame}
\frametitle{Control Variables}
\itemize
\item percentage of the population that lives in urban areas (The World Bank)
\item aggregated civilization identification (based on Samuel Huntington's civilizational) \item colonial heritage - dichotomous, indicates whether a country was colonized (from Valerie Hudson and Donna Lee Bowen's scale in their forthcoming book)
\item percentage of land that is arable (The World Bank)
\item number of unique land neighbors (Wikipedia)
\item level of ethnic fractionalization (Alesina et al., 2003)
\item level of religious fractionalization (Alesina et al., 2003).
\end{frame}

\begin{frame}
\frametitle{Standardizing Variables for Analysis}
\begin{itemize}
\item Standardization of independent variables following the methods put forward in Gelman et al. (2014). 
\item Scale all to have mean=0, standard deviation=0.5
\item Dichotomous variable (colonization status): find proportion of 0's (0.13) and proportion of 1's (0.87), then redefining the 0's as 0.87 and the 1's as -0.13.
\item Categorical variable (civilization): create 3 dichotomous variables which indicated 3 of the levels, then standardized each of those in the same way as colonial status
\end{itemize}
\end{frame}


\begin{frame}
\frametitle{Environmental Indicators}
\footnotesize
\begin{columns}
	\begin{column}{0.40\textwidth}
\itemize
\item Water and Environmental Wellbeing
\item Air Pollution
\item Biodiversity and Pest Regulation
\item Deaths caused by outdoor air pollution
\item Air Quality
\item Global Climate Risk Index
	\end{column}
	\begin{column}{0.55\textwidth}
\centering	
\begin{figure}
  \centering
  \textbf{Average Syndrome Score for Good and Bad Levels of Each Variable}\par\medskip
  \includegraphics[height=4cm,trim={4.2cm 4cm 2cm 4.2cm},clip]{barplotillust.pdf}
  \label{SynPerform}
\end{figure}
	\end{column}
\end{columns}
	Each variable is dichotomized where 0 indicates ``good" levels and 1 indicates ``bad" levels
\end{frame}

\begin{frame}
\frametitle{Preliminary Model}
Hierarchical Bayesian Model
\footnotesize
\begin{equation}
\begin{split}
&\text{Level 1: } y_{im}|p_{im} \sim\ \text{Bernoulli}(p_{im}), \\
& \text{  logit}(p_{im}) = \beta_{m0} + \sum_{j=1}^{7} x_{imj} \beta_{mj} + \sum_{k=1}^{3} x_{imk} \alpha{mk} + \epsilon_{im} \\
&\text{Level 2:   } \beta_0 \sim\ \text{Cauchy}(0,\sigma) \text{, } \beta_{j} \sim\ \text{Cauchy}(0,\theta) \text{, } \alpha_k \sim\ \text{Cauchy}(\mu,\phi) \\
&\text{Level 3:  } \sigma \sim\ \text{Uniform}(5,15), \theta \sim\ \text{Normal}^+(2.5,1), \mu \sim\ \text{Gamma}(3,.5)
\end{split}
\end{equation}
\small
\itemize
\item Not enough data to estimate the parameters on multiple levels
\item Covariance matrix of the draws is not positive definite
\end{frame}

\begin{frame}
\frametitle{The Model}
\begin{equation}
\begin{split}
&y_{im}|p_{im} \sim\ \text{Bernoulli}(p_{im}),  \\
&\text{  logit}(p_{im}) = \beta_{m0} + \sum_{j=1}^{7} x_{imj} \beta_{mj} + \sum_{k=1}^{3} x_{imk} \alpha{mk} + \epsilon_{im} \\
&\text{    } \beta_0 \sim\ \text{Cauchy}(0,\sigma) \text{, } \beta_{j} \sim\ \text{Cauchy}(0,\theta) \text{, } \alpha_k \sim\ \text{Cauchy}(\mu,\phi) \\
&\text{   } \sigma=5, \theta=2.5, \mu=\phi=2
\end{split}
\end{equation}
$i=1,...,n$ corresponds with the country (or observation), $m=1,...,6$ corresponds with the regression model (one model for each environmental variable), $j=1,...,7$ and $k=1,...,3$ corresponds with the independent variables.
\end{frame}

\begin{frame}
\frametitle{Prior Justification}
Cauchy distribution chosen over the normal or the t distributions because the Cauchy is more flexible with extreme values. 
\itemize
\item $\beta$ priors are non-informative
\item $\alpha$ priors correspond with the variables the I have more prior information for: Terrain, Urbanization, and Syndrome
\end{frame}

\begin{frame}
\frametitle{Posterior Distribution}
The posterior distribution is given, up to proportationality:
\begin{equation}
\begin{split}
&p(. | y) \propto 
p(\beta_0,\beta_1,...,\beta_7,\alpha_1,...,\alpha_3,\sigma,\theta,\mu, y) \propto  \\
&\left[\prod_{i=1}^{n} \left(\frac{e^{X_i\beta}}{1+e^{X_i\beta}}\right)^{y_i} \left(\frac{1}{1+e^{X_i\beta}}\right)^{1-y_i}\right] \left[\frac{1}{\pi\sigma\left[1+\left(\beta_0/\sigma\right)^2\right]}\right] \\
&\left[\prod_{j=1}^{6}\frac{1}{\pi\theta\left[1+\left(\beta_i/\theta\right)^2\right]}\right]
\left[\prod_{k=1}^{4}\frac{1}{\pi\phi\left[1+\left((\alpha_i-\mu)/{\phi}\right)^2\right]}\right] 
\end{split}
\end{equation}
\end{frame}

\begin{frame}
\frametitle{Computational Methods: Metropolis-Hastings}
Markov chain Monte Carlo with multivariate proposal scheme:
\begin{itemize}
\item Set initial values for the parameters.
\item Use normal approximation to obtain the hessian estimate for the covariance matrix ($\mathbf{S}$) and the optimized parameter estimates for the mean vector ($\mathbf{m}$).
\item For each iteration: 
\begin{enumerate}
\item Set the proposal: $\mathbf{P} = \mathbf{m} + (\text{chol}(\mathbf{S}))'\text{rnorm}(11)$.
\item Use a Metropolis-Hastings update to determine whether to keep the last samples or update with the proposal, using the proposal density: $f(\mathbf{P}) = -0.5(\mathbf{P}-\mathbf{m})'\mathbf{S}^{-1}(\mathbf{P}-\mathbf{m})$.
\end{enumerate}
\end{itemize}
\end{frame}

\begin{frame}
\frametitle{Computational Methods: Metropolis-Hastings}
\itemize
\item 3 chains, 60,000 draws each
\item initial values: all 0, mean estimates from priors, frequentist estimates
\item burn-in: 20,000, thinning: every fifth draw
\item CPU: under 5 seconds for each chain
\end{frame}

\begin{frame}
\frametitle{Model Checks}
\begin{figure}
\centering
\includegraphics[height=6cm]{conv.png}
\label{fig:diagSyn}
\end{figure}
Posterior draws for Syndrome coefficient for each regression model
\end{frame}

\begin{frame}
\frametitle{Model Checks}
\centering
\textbf{Diagnostics for Syndrome coefficient}
\tiny
\begin{table}[htb]
    \centering
    \begin{tabular}{l|r|r|r}
      & \multicolumn{3}{l}{\textbf{Syndrome Coefficient Diagnostics}} \\
     \hline
     \textbf{Model} & Acceptance Rate & Effective Sample Size & $\hat{R}$ \\
    \hline
    Water and Environmental Wellbeing & 0.34 & 53.58 & 2.77 \\
    Air Pollution  & 0.34 & 2615.55 & 1.10 \\
    Biodiversity and Pest Regulation  & 0.04 & 666.72 & 1.12 \\
    Deaths Caused by Outdoor Air Pollution  & 0.10 & 667.66 & 1.31 \\
    Air Quality & 0.09 & 357.73 & 1.67 \\
    Global Climate Risk Index  & 0.01 & 325.19 & 1.07 \\
    \end{tabular}
    \label{diagSyn}
\end{table}
\end{frame}

\begin{frame}
\frametitle{Syndrome Estimate Results}
\begin{figure}
\textbf{Syndrome Coefficient Estimates and Credible Intervals for each Environmental Model}\par\medskip
\centering
  \includegraphics[height=6cm,trim={0cm 2cm 2cm 3.8cm},clip]{coefficientillust.pdf}
\end{figure}
\end{frame}

\begin{frame}
\frametitle{Frequentist Approach}
\begin{figure}
\textbf{Comparison of Syndrome Estimates in Bayesian and Frequentist Analysis}\par\medskip
\centering
  \includegraphics[height=6cm]{fcoefficientillust.pdf}
\end{figure}
\end{frame}

\begin{frame}
\frametitle{Model Sensitivity}
\begin{equation}
\begin{split}
& \text{\bf{Less Informative Model}} \\
&\text{    } \beta_0 \sim\ \text{Cauchy}(0,\sigma) \text{, } \beta_{j} \sim\ \text{Cauchy}(0,\sigma) \text{, } \alpha_k \sim\ \text{Cauchy}(0,\sigma) \\
&\text{   } \sigma=5
\end{split}
\end{equation}
\begin{equation}
\begin{split}
& \text{\bf{More Informative Model}} \\
&\text{    } \beta_0 \sim\ \text{Normal}(0,\sigma) \text{, } \beta_{j} \sim\ \text{Normal}(0,\theta) \text{, } \alpha_k \sim\ \text{Normal}(\mu,\phi) \\
&\text{   } \sigma=5, \theta=2.5, \mu=\phi=2
\end{split}
\end{equation}
\end{frame}

\begin{frame}
\frametitle{Model Sensitivity Results}
\begin{figure}
\centering
\textbf{Comparison of Syndrome Estimates In Sensitivity Analyses}\par\medskip
\centering
  \includegraphics[height=4cm]{coefficientsensillust.pdf}
  \label{SynCoefEstSens}
\end{figure}
\end{frame}

\begin{frame}
\frametitle{Model Sensitivity Diagnostics}
\begin{table}[htb]
       \tiny
    \centering
\textbf{\footnotesize{Diagnostic Comparisons for Syndrome coefficient in Sensitivity Analyses:}\\ \tiny{ M1=Chosen Model, M2=Less Informative Model, M3=More Informative Model}}\par\medskip
    \begin{tabular}{l||r|r|r||r|r|r||r|r|r}
      & \multicolumn{3}{l}{\textbf{Acceptance Rate}} & \multicolumn{3}{l}{\textbf{Effective Sample Size}} & \multicolumn{3}{l}{\textbf{R-hat}} \\
     \hline
     \textbf{Model} & M1 & M2 & M3 & M1 & M2 & M3 & M1 & M2 & M3 \\
    \hline
    Water \& Env. Wellbeing & 0.34 & 0.35 & 0.34
    & \bf{53.58}  & 22.33 & 22.52
    & 3.36  & \bf{1.00} & 2.13 \\
    Air Pollution  & 0.34  & 0.35 & 0.29
    & \bf{2615.55}  & 22.73 & 123.23
    & \bf{1.32}  & 5.00 & 3.73 \\
    Biodiv. and Pest Reg.  & 0.04 & 0.04 &  0.04
    & \bf{666.72}  & 474.21 & 426.77
    & \bf{1.15}  & 1.39 & 1.45 \\
    Deaths from O.A.P.  & 0.10  & 0.11 & 0.11
    & 667.66  & 429.87 & \bf{2120.84}
    & \bf{1.03}  & 1.37 & 1.05 \\
    Air Quality & 0.09  & 0.10 & 0.09
    & 357.73  & 112.37 & \bf{577.40}
    & \bf{1.01}  & 1.55 & 1.05 \\
    GCRI  & 0.01  & 0.02 & 0.01
    & 325.19  & 199.05 & \bf{407.38}
    & 1.42  & 2.24 & \bf{1.26} \\
    \end{tabular}
    \label{diagSynSens}
\end{table}
\end{frame}

\setbeamercolor{title}{fg=violet}
\begin{frame}
\frametitle{Discussion and Conclusion}
\begin{itemize}
\item Results confirm the hypothesis 
\item The results for Global Climate Risk Index model should be interpreted more cautiously
\item Future research: collect more data for hierarchical approach, improve convergence and model fit
\end{itemize}
Overall, the results of this analysis indicate that the subordination of women in the household is a significant determinant of a country's environmental performance. 
\end{frame}


\end{document}